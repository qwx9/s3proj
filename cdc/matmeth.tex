\section{Matériel et méthodes}

\subsection{Besoins fonctionnels principaux}

L'objectif principal est de produire une interface web fonctionnelle pour les outils \textit{Bio++},
et de la déployer sur un serveur du laboratoire d'accueil accessible depuis l'extérieur.
L'interface doit impérativement être simple et accessible pour un public non-expert,
ciblé essentiellement sur des biologistes travaillant sur des projets en génétique évolutive.
En outre, ce projet permet à un utilisateur potentiel
de s'abstraire de l'installation des différents outils,
qui nécessite une connaissance des environnements Unix-like, de 
leur administration, et de la compilation de logiciels C++.
En d'autres termes, l'idée de ce projet
est de permettre de rendre la suite accessible
à un public beaucoup plus large
en baissant significativement l'effort nécessaire
pour se l'approprier.

Pour ce faire, l'interface doit pouvoir faire l'intermédiaire entre un utilisateur
et la suite \textit{Bio++},
afin de pouvoir spécifier un modèle phylogénétique et rentrer des données,
de façon transparente par rapport à l'implémentation des outils.

En effet, la suite d'outils est imposante en premier abord,
nécessitant l'apprentissage d'un langage déclaratif spécifique à \textit{bppsuite},
l'utilisation et le paramétrage des différents logiciels de la suite,
en passant par l'entrée des données et la récupération des sorties,
ainsi que leur interprétation.


\subsection{Considérations additionnelles}

L'interface doit rendre la suite plus attractive à de nouveaux utilisateurs.
Ainsi, sa bonne performance est très importante.
Sa réactivité, efficacité et la rapidité du rendu
doivent donc être bien prises en compte dans son design et implémentation.

L'introduction à la suite \textit{Bio++} par le biais de la nouvelle interface
sera facilitée par l'accès à des templates de modèles évolutifs,
et un tutoriel rapide,
tous transposés depuis des documents et exemples
déjà disponibles dans les dépôts de \textit{Bio++}.
Leur présence sur l'interface permettra de bien comprendre par l'exemple
les possibilités de la suite.
Un accès au reste de la documentation déjà générée permettra de plus
à des utilisateurs expérimentés d'aller plus loin dans leurs investigations,
en passant par des modèles de plus en plus complexes.


\subsection{Pipeline basique}

Dans son utilisation la plus basique,
\textit{bppml} prend en entrée une séquence alignée dans un alphabet nucléique,
un arbre phylogénétique construit au préalable,
et un modèle avec ses paramètres
avec leurs valeurs initiales.
L'arbre est complet et topologique,
et n'est pas optimisé par le logiciel:
c'est le processus évolutif qui l'est.
Les séquences sont doivent être propres et alignées au préalable.

Les paramètres spécifiés du modèle sont estimés par maximum de vraisemblance
pour inférer le meilleur modèle étant donné l'arbre et les séquences.
Les sorties comprennent les paramètres inférés du modèle,
la meilleure phylogénie reconstruite,
et des fichiers intermédiaires.
Ces fichiers peuvent être réutilisés ensuite
avec d'autres logiciels de la suite \textit{Bio++}
pour d'autres calculs.

La complexité du projet provient du fait que les données d'entrée
peuvent être toutes les combinaisons imaginables
de séquences, de modèles, d'arbres, de paramètres ou de racines d'arbres,
de nombre différent à chaque fois.
La combinaison des différents modèles,
qui peuvent être appliqués sur un arbre ou sur des branches individuelles,
des racines d'arbres,
avec des taux évolutifs différents ou pas, etc.,
est nommée "processus" évolutifs,
qui peuvent eux-même être emboîtés.
Les modèles peuvent être réutilisés
dans un autre run de \textit{bppml}.
Des alphabets autres que nucléiques (protéique, codons) sont gérés par \textit{bppml}, ce qui constitue un avantage.
Les processus sont ensuite appliqués sur les séquences en entrée.


\subsection{Design}

Pour répondre aux besoins,
le design considéré est celui d'une architecture serveur-client classique.

La partie client sera divisée en deux composantes,
l'une gérant l'entrée et la validation des données et paramètres,
et l'autre, le rendu des résultats et leur visualisation.

La partie serveur communiquera avec la machine client
pour générer les lignes de commandes et fichiers d'entrée
pour lancer \textit{bppml},
puis renvoyer les résultats du calcul à l'utilisateur (figure \ref{fig:des}).

Le déploiement de l'outil devra être facile et rapide
pour prendre en compte des changements de serveur
ou d'organisation du système de fichiers différente dans l'avenir.
Sa prise en charge par une autre équipe après le terme du projet
doit être sans effort.
Les dépendances du projet doivent donc en outre
être faciles à gérer.
L'implémentation ne doit pas utiliser des chemins en dur sur le système de fichiers
mais utiliser des fichiers de configuration
pour s'abstraire de sa localisation.

En termes de performance,
le temps de réponse du serveur doit être minimal,
sans temps de latence supplémentaire par rapport au temps de calcul de la suite.
De l'autre côté, la partie client doit être peu lourde
pour conserver une rapidité même sur des ordinateurs peu puissants,
ce qui est le cas assez fréquemment.

La suite \textit{bppml} peut prendre en charge
des types de données multiples, un ou plusieurs sous-modèles,
avec potentiellement des paramètres différents,
et une multitude de combinaisons
qui rend le nombre de possibilités rapidement ingérable.
Le programme initial sera donc très simple mais modulaire et extensible,
pour implémenter les fonctionnalités basiques,
qui vont soutenir le reste de l'interface,
très rapidement.
Au début, l'interface ne gérera que
des combinaisons d'entrées et de modèles très simples,
puis de plus en plus de possibilités seront ajoutées de façon itérative.
Ceci permettra
de tester facilement au fur et à mesure les nouvelles fonctionnalités,
de permettre de limiter l'implémentation dans le cadre de ce projet
aux fonctionnalités principales,
de donner la capacité d'ajouter le reste par la suite,
et de s'adapter à l'évolution future de la suite.

\begin{figure}
	\caption{Design général de \textit{bpp-web}}
	\includegraphics[scale=0.5]{fig/SchemaConcept.pdf}
	\centering
	\label{fig:des}
\end{figure}


\subsection{Implémentation}

L'interface sera basée sur une installation de Linux,
adaptée au serveur disponible hébergé par le laboratoire d'accueil.

Elle devra à terme pouvoir gérer tous les types
et combinaisons de données d'entrée qu'accepte \textit{bppml}.
Des templates de runs entiers seront mis à disposition
pour pouvoir réutiliser des modèles adaptés à des situations
plus ou moins typiques ou complexes,
ou même juste en tant qu'exemple.

Des séquences pourront être entrées soit directement sur un champ textuel,
soit en soumettant un fichier,
par le biais d'un dialogue ou par drag-and-drop.
Des arbres phylogénétiques pourront être soumis par le même mécanisme.
Un nouvel élément de ce type pourra être rajouté et rempli de la même façon.

Les modèles pourront
soit être rentrés directement dans la syntaxe reconnue par \textit{bppml},
par exemple si les résultats d'un run sont réutilisés,
ou alors en utilisant un mécanisme itératif pour construire un modèle interactivement.
Il sera basé sur des éléments typiques et bien connus,
comme des menus déroulants pour des options fréquentes,
des champs de recherche si les options sont trop nombreuses,
et des boutons pour modifier le nombre d'éléments ou les modifier.
Le modèle courant pourrait avoir un support visuel
qui pourrait être édité lui-même,
pour ajouter, supprimer, ou masquer des éléments,
ainsi que d'autres manipulations.
De plus, des templates de modèles seront mis à disposition,
pour construire des modèles à partir d'exemples fréquents,
et pour copier ou sauvegarder des modèles construits sur l'interface.

Les processus évolutifs globaux appliqués sur les séquences
seront constitués de toutes ces parties, en dehors des séquences.
Les sorties seront une phylogénie optimisée
avec ses fichiers correspondants
(intermédiaires, logs, résultats)
qui pourront être téléchargés.
La phylogénie elle-même sera représentée avec des graphiques interactifs.

Les logiciels ne dépendent pas de données volumineuses
ou de connexion à une base de données,
donc la gestion de l'espace de disque dur ne sera pas conséquente.
Par contre, \textit{bppml} peut,
pour des modèles relativement complexes,
prendre beaucoup de temps de calcul et de mémoire vive.
Les résultats devront ainsi potentiellement
être rendus en différé,
par exemple par notification par e-mail,
et en fournissant un lien unique vers eux.
La page résultats sera obtenue en suivant ce lien.

Le serveur peut être implémenté sous Python avec Flask.
Le client utilisera Typescript transpilé en javascript.
Les sorties attendues sont assez simples,
pour ne nécessiter que des visualisations relativement simples
sur graphiques SVG générées par D3.js.
Une implémentation single-page dynamique est souhaitée.

L'aspect visuel sera géré par des fichiers CSS.

Les pages exposées à l'utilisateur seront clairement documentées,
soit sur une page dédiée,
soit directement et de façon non-intrusive sur l'interface.
De plus, des tool-tips sur les différentes parties de l'interface
la rendront plus facilement utilisable.
Une page indiquera les auteurs, des contacts,
des liens vers les dépôts et leur site officiel,
et les articles de recherche sur la suite.

Le développement se fera en local sur machines personnelles
avant de le déployer pour qu'il soit accessible publiquement,
et ainsi avoir directement un site fonctionnel dès publication.


\subsection{Extensions possibles}

L'interface sera implémentée en anglais durant le projet,
mais à terme, elle pourrait être traduite dans d'autres langues.

La fonctionnalité de l'interface est primordiale,
mais un travail sur son aspect visuel est souhaitable également.
