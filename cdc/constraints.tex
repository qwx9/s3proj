\section{Contraintes et difficultés}

\subsection{Coûts}

Aucun coût financier n'est associé au projet.
Le serveur web sur lequel le site sera déployé est déjà fourni par le laboratoire fourni.
Les dépendances de la suite \textit{Bio++} et du site sont gratuites et open source,
et disponibles sur \textit{github.com}.
Les exemples de modèles et de données permettront facilement
de tester de développer le site sur machine personnelle.


\subsection{Délais}

Le délai pour rendre le délivrable final est relativement court,
d'environ 10 semaines.
Uniquement 2 jours par semaine sont entièrement dédiées au projet,
le reste comportant demi-journées ou journées entières de cours.

Le suivi de l'enseignement et la préparation
aux examens dès fin novembre
nécessite une organisation efficace
un travail régulier,
et une bonne communication entre les membres du projet
et les responsables pédagogique et de stage.


\subsection{Autres contraintes}

Un problème potentiel auxquels des solutions doivent être prévues à l'avance
est un reconfinement possible dû au COVID-19,
qui pourrait entraver la communication entre les membres
et surtout avec les responsables du projet.

\textbf{
Des dispositifs de l'université,
dont des salles de travail disponibles aux étudiants du master bioinformatique,
et des systèmes de visioconférence,
pourraient aider à pallier à cette contrainte.
}

L'inexperience des membres du projet en phylogénomique
et en l'implémentation de sites web,
nécessitera un apprentissage rapide et efficace
des technologies et pratiques courantes.

\textbf{\\
- developpement iteratif\\
\\
- respect de programming standards rigoureux\\
\\
- tests a chaque etape\\
\\
- coordination et review entre collaborateurs, nott pour bugs\\
}

L'évolution constante de \textit{Bio++}
laisse la possibilité de survenue de bugs
cassant aussi la fonctionnalité du site.
