\section{Introduction}

\subsection{Contexte}
	Dans tous les domaines de la biologie de nouveaux programmes sont développés au quotidien. Ces programmes sont en constante évolution et de plus en plus complexes. Avec cette complexité grandissante, ces outils peuvent se retrouver en pratique peu utilisés par de nombreux utilisateurs potentiels. En effet, l'installation ou l'appel de programmes via la ligne de commande peut vite être rédhibitoire.
	La montée du cloud computing et des solutions no code accessibles par des interfaces web illustre ce mouvement de démocratisation des outils en les rendant plus facile d'utilisation, sans installation locale.
	L'objectif de ce projet est de développer une interface web permettant l'utilisation de \textit{bppsuite} dont le paramétrage de l'appel peut devenir très lourd.
	Le projet est effectué avec le LBBE, en collaboration avec Laurent Guéguen du groupe cocon, équipe PEGASE.
	
	
\subsection{Existant}
	\textit{Bio++}\footnote{https://github.com/BioPP} ~\cite{Guéguen} est une suite de librairies en C++ dédiées à la phylogénie et à l'analyse d'évolution de séquences biologiques. Au sein de cette suite, bppsuite est une solution stand-alone qui permet l'utilisation de libraries de \textit{Bio++} (\textit{bpp-core, bpp-seq, bpp-phyl, bpp-popgen}) dont le but est de simuler l'évolution de séquences, des analyses phylogéniques, l'estimation du meilleur modèle d'évolution... Cette suite manipule des objets variées tels que des arbres, des séquences, des modèles, des process dont les possibilités combinatoires sont infinies. Voici la liste des modules de \textit{Bio++} sur lesquels nous allons travailler:
%itemize?
	
	\subsubsection*{bpp-seq}
	Cette librairie est composée de classes qui stockent les séquences biologiques. Ces séquences peuvent être de l'ADN, des codons, de l'ARN ou des acides aminés. Cette librairie supporte aussi les séquences alignées.
	
	\subsubsection*{bpp-phyl}
	Cette librarie contient les classes et les méthodes pour étudier l'évolution phylogénique et moléculaire. C'est elle qui définit la classe Tree (plus complexe dans la branche newlik en développement) qui est une liste de noeuds. Un noeud contient les méthodes permettant d'accéder aux noeuds père et fils ainsi que d'autre information comme sont nom, la longueur de la branche vers le noeud père...
	Elle contient aussi les outils permettant de reconstruire un arbre basé sur l'évolution des séquences. Cette reconstruction peut se faire avec différentes méthodes : maximum de parcimonie, par distance, par vraisemblance...
	De nombreux modèles évolutifs sont aussi fournis avec cette librarie. Tous les modèles classiques de substitution pour les nucléotides ou les protéines sont inclus, de plus l'implémentation de modèles est possible.
	\subsubsection*{bpp-popgen}
	Cette librairie a été conçue pour l'étude de génétique de population. Elle se base sur \textit{bpp-seq} et permet de manipuler des jeux de séquences ainsi que de calculer différentes statistiques comme le polymorphisme, l'hétérozygotie...
	
	\subsubsection*{bpp-core}
	Cette librarie définie tous les objets de base (string, numbers, vectors) utilisés par les librairies qui composent \textit{Bio++}.
	
	\subsubsection*{bppsuite}
	Toutes les libraries citées précédemment sont indépendantes les unes des autres. Le rôle de \textit{bppsuite} est d'offrir des utilitaires prêts à l'emploi qui permettent de combiner tous ces modules afin de mener des analyses complexes avec une syntaxe commune. En combinant des jeux de données grands, avec une multitude d'arbres et de modèles différents les possibilités d'appel de \textit{bppsuite} sont nombreuses et complexes. Le paramétrage de \textit{bppsuite} peut se faire via la ligne de commande mais en pratique elle est plutôt faite via un fichier texte de configuration. Ce fichier définie les objets à manipuler et les paramètres qui leur sont associés.
	Un des composant de \textit{bppsuite} est \textit{bppml} qui applique les principes de maximum de vraissemblance à la phylogénie. C'est cette partie de \textit{bppsuite} sur laquelle nous travaillerons.
	
	
\subsection{Objectifs}
	%Interface web, génération du fichier d'appel
	%Recuperation des fichiers résultats
	%Aller retour entre config avec GUI et fichier text de généré
	%au long terme de runner sur un serveur
	La complexité grandissante du fichier de paramétrage, associé à une grammaire propre peut être rédhibitoire à l'utilisation de \textit{bppml}. L'objectif de ce projet est de démocratiser l'utilisation de cet outil en développant une interface web intuitive qui permettra à l'utilisateur de réaliser la modélisation qu'il désire et retournera le fichier de configuration. Cette interface devra pouvoir jongler entre le paramétrage de \textit{bppml} et une représentation visuelle de la modélisation. Cette modélisation sera ensuite exécutée par \textit{bppml} et les fichiers de sorties devront aussi pouvoir être accessible par l'interface. Au long terme, le but est de faire tourner \textit{bppml} sur un serveur situé sur le campus, offrant aux utilisateurs curieux une version sans code et installation afin de pouvoir facilement utiliser ces outils. Au long terme, les autres parties de \textit{bppsuite} pourront être ajouter à cette implémentation appelée \textbf{bpp-web}.
	
\subsection{Acceptabilité}
	%Interface graphique
	%implémentation du plus de modèle possible (exhaustivité)
	% préparation du terrain pour le calcul sur serveur
	Les critères principaux sont une interface graphique constituée d'icônes permettant de déclarer et paramétrer les différents objets de façon intuitive avec une représentation visuelle de la modélisation en cours d'élaboration. Cette interface web doit être flexible pour pouvoir facilement implémenter les nouvelles fonctionnalités de \textit{Bio++} qui est en constante évolution ainsi qu'ajouter des modèle mis au point par l'utilisateur. L'interface doit être facilement implémentable sur un serveur pour le développement à long terme du projet.
Enfin une documentation précise devra être livrée avec le projet aussi bien pour les utilisateurs que pour de potentiels développeurs
 afin de rendre l'implémentation de nouveaux modules de \textit{bppsuite} aussi facile que possible.
	
